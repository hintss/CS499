\section{Functionality}
%Functionality
%What functionality does the module provide?
ZFS in our identity management platform provides numerous vital functions, the most obvious being storage for students, staff and faculty. Every currently affiliated person on campus has their own space on ZFS, accessible from machines on campus as well as their own computers. Although ZFS is natively a *nix-born file system (first introduced with OpenSolaris in 2005), it is made available to campus users through Samba in order to be accessible from all operating systems: MS Windows, OS X, and Linux. 

ZFS on campus also has a ZFSWeb component. Every currently affiliated human on campus has a directory made to be world-accessible through HTTP. Its outside address is in the form of http://www.cpp.edu/$\sim$[uid]/. Up until recently, this was a non-optional feature, however, now it must be enabled through the identity management front-end on CalPoly website.

ZFS as a file system has numerous important features, such as access list support and high limit of volume and file size (256 zebibytes and 16 exbibytes, relatively). Those are important for a file system used for campus-wide user shares. Access lists allow picking and choosing access to files and directories on a per-person or a per-group level, while the high volume size limit means we won't run into the software space limitation any time soon. 
